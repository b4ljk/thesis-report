Бакалаврын судалгааны тайлан бол шинжлэх ухаан, инженерчлэлийн бүтээлийг тайлагнах баримт бичиг юм. Мэдээлэл, компьютерийн ухааны тэнхимд хийгддэг ажлууд нь гол төлөв судалгаа, эрдэм шинжилгээний ажил эсвэл мэдээллийн технологийн инженерчлэлийн чиглэлийнх байдаг. Эрдэм шинжилгээний ажил бичиж байгаа бол судалгааг хийх сэдвийн судлагдсан байдал, боловсруулсан шийдэл, арга аргачлал, туршилт, үр дүнгийн боловсруулалт зэрэг агуулга зонхилно. Харин инженерчлэлийн чиглэлийн ажлын хувьд хэрэглэгчийн шаардлага, шинжилгээ, зохиомж, хэрэгжүүлэлт гэсэн агуулгыг голлож бичдэг.

\section{Тайлангийн бүтэц ба төлөвлөлт}
Тайланг бичиж эхлэхээс өмнө тайлангийн бүтцийг тодорхойлж төлөвлөх хэрэгтэй. Тайлан хэдэн бүлгээс тогтох, аль бүлэгт ажлын аль хэсгийг харуулах, хоорондоо хэрхэн уялдаатай байх, бүлгийн дэд гарчиг, дэд гарчиг доторх цогцолбороор юу өгүүлэх зэргийг урьдчилж боловсруулвал тайлан бичихэд илүү хялбар болдог. Энэ нь тайланг бичиж эхлэхээс өмнө ямар баримт бичиг болохыг бүхэлд нь харах боломжийг олгох бөгөөд тайлангийн цар хүрээг тодорхойлж буй хэрэг юм. Доорх жишээнд бүлгийг нэрлэж, бүлэг болон дэд бүлэг ямар агуулга байгааг хэрхэн төлөвлөж байгааг харуулав.

\begin{itemize}
	\item Бүлэг 1. Үгийн утга зүйн цахим сан
	\textit{[Үгийн утга зүй, цахим сангийн хэрэглээ, бүтэц, агуулга зэргийг товч бичих]}
	\begin{itemize}
	    \item 1.1 Нутгийн мэдлэгийн цөм \textit{[НМЦ гэж юу болох, бүрэлдэхүүн хэсэг, ойлголтын цөм, хэлний цөмийн бүтэц, түүнд агуулагдах элементүүдийг тайлбарлах]}
	    \item 1.2 Холбоотой ажлууд \textit{[Үгийн утга зүйн цахим сангийн талаарх бидний ажилтай нягт холбоотой ажлуудыг товч танилцуулж бичих, мөн бидний ажилтай холбоотой онцлог шинж, ялгарах талуудыг дурдах]}
	    \item 1.3 Цахим сан үүсгэх асуудлууд \textit{[Дээрх холбоотой ажлуудад тулгарч байгаа нийтлэг асуудлууд, тэдгээр ажлуудад авч үзээгүй зүйлс зэргийг бичих]}
	\end{itemize}
	\item Бүлэг 2. Нутагшуулах аргачлал
	\item ...
\end{itemize}

\subsection{Тайлангийн гарчиг}
Тайлангийн гарчиг нь илүү явцуу, тухайн ажлыг бүхэлд нь илэрхийлж чадахуйц байхаар өгдөг. Эрдэм шинжилгээ судалгааны ажлын хувьд [Асуудал][Арга/Шийдэл][Ай/Сэдэв] гэсэн хэсгүүдийг агуулсан байвал илүү тодорхой болдог. Жишээ нь, \textit{"Нутгийн Мэдлэгийн Цөмийг хамтын ажиллагаат олны хүчээр үүсгэх аргачлал ба хэрэгжүүлэлт"} сэдвийн хувьд \textit{"үүсгэх аргачлал ба хэрэгжүүлэлт"} гэдэг нь асуудал, \textit{"олны хүчээр"} гэдэг нь арга, шийдэл, \textit{"Мэдлэгийн цөм"} гэдэг нь хэдий оноосон нэр боловч knowledge base, knowledge core гэх англи утгыг агуулж байгаа тул ай (domain) буюу сэдвийг тодорхой хэмжээнд илэрхийлж байна. Дээрх гурван хэсгийн дарааллын хувьд сэдвийн нэрийн найруулгаас хамаарах биз ээ.
\\Програм хангамж хөгжүүлэлтийн ажлын хувьд [програмын нэр][төрөл][гүйцэтгэсэн ажил] зэрэг агуулгыг тайлангийн гарчигтаа оруулвал илүү тодорхой болж бусад ажлуудаас ялгарч өгдөг. Програмын нэр нь ерийн эсвэл оноосон нэртэй байж болох юм. 
Англи нэрээ ч мөн адил удирдагч багштайгаа сайтар ярилцаж зөвлөсний үр дүнд өгөх нь зүйтэй. Дипломын ажлын нэр сургууль төгссөнийг гэрчлэх дипломын хавсралт дээр бичигддэг учир тун ач холбогдолтой хандах хэрэгтэй.

\subsection{Бүлэг нэрлэх}
Бүлгийг нэрлэхдээ аль болох тухайн ажилтай холбоотой нэр томъёо, үгийг ашиглах нь оновчтой байдаг. Энэ нь тухайн ажлыг бусад ажлаас ялгах, тайлангийн агуулгыг ерөнхийд нь ойлгох боломжийг уншигчдад олгодог. Сэдвийн судалгаа, шинжилгээ ба зохиомж, хэрэгжүүлэлт гэх мэтээр ерөнхий нэрлэснээс илүү ойлгомжтой болгодог тул аль болох бүлгийн нэрийг оновчтой, бүлэг доторх агуулгадаа тохирсон байдлаар өгөх нь зүйтэй байна. Мөн бүлгийн нэр нь нэр үгээр байвал зохимжтой байдаг. Эсвэл үйлт нэрээр бичдэг.

\section{Удирдтгал бичих}
Удиртгал буюу оршил хэсэгт ажлыг хийх хэрэгцээ, шаардлага, үндэслэл, ажлын зорилго, зорилгод хүрэх зорилтуудыг бичдэг. Мөн тайлангийн бүтэц буюу аль бүлэгт юуны талаар өгүүлснийг бичнэ. Удиртгалыг хэсэг нэгээс хоёр нүүрт багтаан цогцолборууд хоорондоо нягт уялдаатай бичсэн байдаг.

\section{Дүгнэлт бичих}
Дүгнэлт нь бүхэлдээ юу хийж гүйцэтгэж ямар үр дүнд хүрсэн. Энд ажлын онцлог, давуу тал, шинэлэг байдал зэрэг бусад ажлаас ялгарах гол шинжүүдийг бичвэл илүү оновчтой байдаг. Түүнчлэн цаашид хэрэгжүүлэх ажил, нэмж гүйцэтгэх саналыг мөн оруулдаг. Ийм саналыг тухайн ажлыг гүйцэтгэсэн хүн илүү тодорхой хэлж чаддаг бөгөөд ажлын сул тал, гүйцээгүй зүйлсийг зөв тодорхойлж буйг илэрхийлэх юм.


\chapter{Бичвэр боловсруулалт}
Бүлгийн гарчгийн дор тухайн бүлэгт юу агуулж байгаа, юуны талаар өгүүлэхийг товч бичих нь баримт бичгийг уншигчдад илүү ойлгомжтой болгодог.
% Бүлгийн дэд гарчиг
\section{Шинэ мөр ба цогцолбор}
Латекс бичих явцад олон хоосон зай, шинэ мөр авахад гаралтын файлд ганцхан хоосон зайгаар дүрсэлж харуулдгаараа бусад засварлагчаас ялгаатай юм.

Шинэ мөр буюу цогцолбор (paragraph) авахдаа хоёр удаа enter товч дарах буюу нэг хоосон мөр үлдээж бичнэ. \par Эсвэл par командыг бичнэ.
\\Харин шинэ мөр авахдаа хоёр ширхэг ургашаа налуу зураас дарааллуулан бичнэ.  Дэлгэрэнгүйг \cite{pharagraph1}-с унш.

Contrary to popular belief, Lorem Ipsum is not simply random text. It has roots in a piece of classical Latin literature from 45 BC, making it over 2000 years old. Richard McClintock, a Latin professor at Hampden-Sydney College in Virginia, looked up one of the more obscure Latin words, consectetur, from a Lorem Ipsum passage, and going through the cites of the word in classical literature, discovered the undoubtable source. Lorem Ipsum comes from sections 1.10.32 and 1.10.33 of "de Finibus Bonorum et Malorum" (The Extremes of Good and Evil) by Cicero, written in 45 BC. This book is a treatise on the theory of ethics, very popular during the Renaissance. The first line of Lorem Ipsum, "Lorem ipsum dolor sit amet..", comes from a line in section 1.10.32.

The standard chunk of Lorem Ipsum used since the 1500s is reproduced below for those interested. Sections 1.10.32 and 1.10.33 from "de Finibus Bonorum et Malorum" by Cicero are also reproduced in their exact original form, accompanied by English versions from the 1914 translation by H. Rackham.

\section{Бичвэр зэрэгцүүлэх}
\subsection{Зүүн тийш зэрэгцүүлэх}
\begin{flushleft}
	Contrary to popular belief, Lorem Ipsum is not simply random text. It has roots in a piece of classical Latin literature from 45 BC, making it over 2000 years old. Richard McClintock, a Latin professor at Hampden-Sydney College in Virginia, looked up one of the more obscure Latin words, consectetur, from a Lorem Ipsum passage, and going through the cites of the word in classical literature, discovered the undoubtable source. Lorem Ipsum comes from sections 1.10.32 and 1.10.33 of "de Finibus Bonorum et Malorum" (The Extremes of Good and Evil) by Cicero, written in 45 BC. This book is a treatise on the theory of ethics, very popular during the Renaissance. The first line of Lorem Ipsum, "Lorem ipsum dolor sit amet..", comes from a line in section 1.10.32.

	The standard chunk of Lorem Ipsum used since the 1500s is reproduced below for those interested. Sections 1.10.32 and 1.10.33 from "de Finibus Bonorum et Malorum" by Cicero are also reproduced in their exact original form, accompanied by English versions from the 1914 translation by H. Rackham.
\end{flushleft}

\subsection{Баруун тийш зэрэгцүүлэх}
\begin{flushright}
	Contrary to popular belief, Lorem Ipsum is not simply random text. It has roots in a piece of classical Latin literature from 45 BC, making it over 2000 years old. Richard McClintock, a Latin professor at Hampden-Sydney College in Virginia, looked up one of the more obscure Latin words, consectetur, from a Lorem Ipsum passage, and going through the cites of the word in classical literature, discovered the undoubtable source. Lorem Ipsum comes from sections 1.10.32 and 1.10.33 of "de Finibus Bonorum et Malorum" (The Extremes of Good and Evil) by Cicero, written in 45 BC. This book is a treatise on the theory of ethics, very popular during the Renaissance. The first line of Lorem Ipsum, "Lorem ipsum dolor sit amet..", comes from a line in section 1.10.32.

	The standard chunk of Lorem Ipsum used since the 1500s is reproduced below for those interested. Sections 1.10.32 and 1.10.33 from "de Finibus Bonorum et Malorum" by Cicero are also reproduced in their exact original form, accompanied by English versions from the 1914 translation by H. Rackham.
\end{flushright}


\section{Хэлбэржилт}
Энэ бүлэгт бичвэрийг хэлбэржүүлэх (format) командуудын талаар дурьдана. Илүү дэлгэрэнгүйг \cite{format1}-с хар.

\subsection{Тодруулах}
\texttt{\textbackslash textbf} командаар бичвэрийг \textbf{тодруулах буюу болд} болгоно.

\subsection{Налуулах}
\texttt{\textbackslash textit} командаар бичвэрийг \textit{бичмэл буюу италик} болгоно.

\subsection{Доогуур зураас}
\texttt{\textbackslash underline} командаар бичвэрийг \textbf{тодруулах буюу болд} болгоно.

\section{URL оруулах}
\texttt{\textbackslash url} команд дотор холбоосыг бичнэ. \url{http://milab.num.edu.mn}


\section{Жагсаалт}
\subsection{Энгийн жагсаалт}
\texttt{\textbackslash begin\{itemize\}} командын дотор энгийн жагсаалтыг бичнэ \cite{list}.
\begin{itemize}
	\item Жагсаалтын эхний элемент
	\item Жагсаалтын хоёрдугаар элемент
	\item Жагсаалтын гуравдугаар элемент
	\item Жагсаалтын дөрөвдүгээр элемент
\end{itemize}

\subsection{Дугаартай жагсаалт}
\texttt{\textbackslash begin\{enumerate\}} командын дотор энгийн жагсаалтыг бичнэ \cite{list}.
\begin{enumerate}
	\item Жагсаалтын эхний элемент
	\item Жагсаалтын хоёрдугаар элемент
	\item Жагсаалтын гуравдугаар элемент
	\item Жагсаалтын дөрөвдүгээр элемент
\end{enumerate}


\chapter{Ишлэл, зүүлт}
\section{Ишлэл}
Ашигласан материал эсвэл номзүйг бичвэр тодор ишлэхдээ cite командаар заалтыг нь оруулна.
Үүний тулд энэ хуудасны хамгийн доор байгаа \textit{Ашигласан материал, ном зүй} хэсэгт
bibitem командыг нэмнэ. \\


Жишээ нь: bibitem\{image1\} Гарчиг, Зохиогчдын нэр, хэвлэсэн он, хэвлэсэн газар

Дээрх жишээнд image1 гэдэг нь ишлэх нэр. Доод талын мөрөнд нь байгаа дарааллын дагуу
ашигласан материалыг бичнэ.

Ишлэхдээ cite командад ишлэх нэрийг дамжуулж өгнө. Жишээ нь cite\{image1\}.
\section{Зүүлт}
Зүүлтийг footnote командаар оруулна \footnote{Энэ холбоосоос зүүлтийн талаар дэлгэрэнгүй унш: \url{https://www.sharelatex.com/learn/Footnotes}}.

\chapter{Зураг}
Зураг оруулахдаа includegraphics командыг ашиглана. Доорх жишээнд figure01.png гэдэг нь зургийн файлын нэр бөгөөд өргөтгөлийг заавал бичих шаардлагагүй. Зургийн файл нь main.tex файлтай нэг фолдерт байх шаардлагатайг анхаарна уу! Дэлгэрэнгүйг \cite{image1}-с үз.

\includegraphics{figure01.png}


\section{Зургийн хэмжээ өөрчлөх}
Хэмжээг томруулахдаа 0-1 хооронд утга ашиглана. Хэрэв 2 гэвэл 2 дахин томроно.
\begin{center}
	includegraphics[scale=0.5]\{figure01\}
\end{center}

\includegraphics[scale=0.9]{figure01}

Өндөр өргөнийг шууд зааж өгч болох бөгөөд дөрвөлжин хаалтан дотор доорх байдлаар бичнэ.
\begin{center}
	includegraphics[width=3cm, height=4cm]\{figure01\}
\end{center}
\includegraphics[width=3cm, height=4cm]{figure01}

\section{Зураг эргүүлэх}
Зургийн эргүүлэхдээ angle параметрт эргүүлэх өнцгийн хэмжээг өгнө.
\begin{center}
	includegraphics[width=3cm, height=4cm, angle=45]\{figure01\}
\end{center}
\includegraphics[width=3cm, height=4cm, angle=45]{figure01}

\section{Зургийн нэр}
Зургын нэрийг begin\{figure\} хооронд includegraphics командтай хамт оруулна Зураг \ref{fig:lion1}-ыг хар.

Энд зургийн нэрээс гадна label-ийг давхар бичиж өгөх шаардлагатай ба энэ нь зургийн дугаараар заалт хийхэд ашиглана. Жишээ нь: Зураг \ref{fig:lion2}

\begin{figure}[h]
	\centering
	\includegraphics[scale=0.9]{figure01}
	\caption{Зураг голлуулах}
	\label{fig:lion1}
\end{figure}

\section{Зураг голлуулах}
Зургийг голлуулахдаа includegraphics командын өмнө centering
командыг бичээд reflectbox командыг includegraphics болон caption
командуудад үйлчлэхээр оруулна.

\begin{figure}[h]
	\includegraphics[scale=0.5]{figure01.png}
	\caption{Зургийн нэрийг энд бичнэ}

	\label{fig:lion2}
\end{figure}

\section{Зургийн чанар}
LaTex-т зургийг вектор форматаар (svg, eps) оруулбал хэвлэх болон томруулж харахад зургийн чанар
алдагдахгүй. Тиймээс аль болох вектор зураг оруулж өгвөл зүгээр.

\chapter{Хүснэгт оруулах}
Хүснэгт оруулахад tabular командыг ашигладаг \cite{table}.

\begin{table}[h]
	\centering
	\caption{Хүснэгтийн нэр. Хүснэгтийн нэр хүснэгтийн дээд талд байрлана. }
	\label{my-label}
	\begin{tabular}{|l|l|l|l|l|}
		\hline
		\textbf{Багана1} & \textbf{Багана2}  & \textbf{Багана3} & \textbf{Багана4} & \textbf{Багана5} \\ \hline
		өгөгдөл          & \textit{өгөгдөл1} &                  &                  &                  \\ \hline
		                 &                   &                  &                  &                  \\ \hline
		                 &                   &                  &                  &                  \\ \hline
	\end{tabular}
\end{table}

\section{Хүснэгт зурах хэрэгсэл}
Цэвэр LaTex кодоор Хүснэгт үүсгэхэд харьцангуй төвөгтэй байдаг учир
хялбар хэрэгслийг ашиглаж болно.

Тухайлбал \url{https://www.tablesgenerator.com/} холбоосруу орж хүснэгтийг визуал орчинд зураад үүсгэж өгсөн LaTex кодыг энд хуулж оруулна.

\chapter{Код ба алгоритм оруулах}
Код оруулахдаа begin\{lstlisting\}  ... end\{lstlisting\} командын хооронд бичнэ.

\begin{lstlisting}[language=C, caption=С хэлний кодын жишээ, frame=single]
#include <stdio.h>
#define N 10
/* Block
 * comment */
int main()
{
    int i;
    // Line comment.
    puts("Hello world!");
    for (i = 0; i < N; i++)
    {
        puts("LaTeX is also great for  programmers!");
    }
    return 0;
}
\end{lstlisting}

Мөн кодын эх файлыг шууд оруулж ирж болох бөгөөд доорх командыг бичнэ.

\lstinputlisting[language=C, firstline=11, lastline=16, caption=Кодын файлаас хэсэгчилж оруулах]{src/hello.c}

Мэдээллийн технологи, программ хангамжийн ажлын тайланд алгоримтыг хийсвэр кодын бичиглэлээр оруулах шаардлага гардаг. Дараах жишээгээр (Алгоритм \ref{alg:task_gen}) хийсвэр кодоор хэрхэн бичиж болохыг харуулав. Мөн бичвэр дотроо алгоритмд ашиглаж байгаа $parentId$ хувьсагчийг дурдаж бичиж болдог.

\makeatletter
\newenvironment{megaalgorithm}[1][htb]{%
	\renewcommand{\ALG@name}{Алгоритм}% Update algorithm name
	\begin{algorithm}[#1]%
		}{\end{algorithm}}
\makeatother

\begin{megaalgorithm}
	\caption{Даалгавар үүсгэх алгоритм}\label{alg:task_gen}
	\begin{algorithmic}[1]
		\Function{traverse}{$parentId$}\Comment{parentId--эцэг ойлголтын дугаар}
		\State $children \gets \Call{getChildConceptIds}{parentId$}
			\State $childCount \gets children.count$
		\If{$childCount == 0$}
		\State \textbf{return}
		\EndIf
		\For{$i = 0$ \textbf{to} $childCount$}
		\State \Call{generateTask}{$children_i$}\Comment{Орчуулгын даалгавар үүсгэх}
		\EndFor
		\For{$i = 0$ \textbf{to} $childCount$}
		\State \Call{traverse}{$children_i$}
		\EndFor
		\EndFunction
	\end{algorithmic}
\end{megaalgorithm}