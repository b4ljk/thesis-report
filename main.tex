%!TEX TS-program = xelatex
% !TeX program = xelatex
%!TEX encoding = UTF-8 Unicode
%----------------------------------------------------------------------------------------
%   Доорх хэсгийг өөрчлөх шаардлагагүй
%----------------------------------------------------------------------------------------
\documentclass[12pt,A4]{report}

\usepackage{fontspec,xltxtra,xunicode}
\setmainfont[Ligatures=TeX]{Times New Roman}
\setsansfont{Arial}

% \usepackage[utf8x]{inputenc}
% \usepackage[mongolian]{babel}
%\usepackage{natbib}
\usepackage{geometry}
%\usepackage{fancyheadings} fancyheadings is obsolete: replaced by fancyhdr. JL
\usepackage{fancyhdr}
\usepackage{float}
\usepackage{afterpage}
\usepackage{graphicx}
\usepackage{amsmath,amssymb,amsbsy}
\usepackage{dcolumn,array}
\usepackage{tocloft}
\usepackage{dics}
\usepackage{nomencl}
\usepackage{upgreek}
\newcommand{\argmin}{\arg\!\min}
\usepackage{mathtools}
\usepackage[hidelinks]{hyperref}

\usepackage{algorithm}
\usepackage{algpseudocode}

\usepackage{listings}
\DeclarePairedDelimiter\abs{\lvert}{\rvert}%
\makeatletter
\usepackage{caption}
\captionsetup[table]{belowskip=0.5pt}
\usepackage{subfiles}

\usepackage{listings}

\usepackage{color}
\definecolor{lightgray}{rgb}{.9,.9,.9}
\definecolor{darkgray}{rgb}{.4,.4,.4}
\definecolor{purple}{rgb}{0.65, 0.12, 0.82}

\lstdefinelanguage{JavaScript}{
  keywords={typeof, new, true, false, catch, function, return, null, catch, switch, var, if, in, while, do, else, case, break},
  keywordstyle=\color{blue}\bfseries,
  ndkeywords={class, export, boolean, throw, implements, import, this},
  ndkeywordstyle=\color{darkgray}\bfseries,
  identifierstyle=\color{black},
  sensitive=false,
  comment=[l]{//},
  morecomment=[s]{/*}{*/},
  commentstyle=\color{purple}\ttfamily,
  stringstyle=\color{red}\ttfamily,
  morestring=[b]',
  morestring=[b]"
}

\lstset{
   language=JavaScript,
   backgroundcolor=\color{lightgray},
   extendedchars=true,
   basicstyle=\footnotesize\ttfamily,
   showstringspaces=false,
   showspaces=false,
   numbers=left,
   numberstyle=\footnotesize,
   numbersep=9pt,
   tabsize=2,
   breaklines=true,
   showtabs=false,
   captionpos=b
}
\renewcommand{\lstlistingname}{Код}
\renewcommand{\lstlistlistingname}{\lstlistingname ын жагсаалт}

\usepackage{color}
\definecolor{codegreen}{rgb}{0,0.6,0}
\definecolor{codegray}{rgb}{0.5,0.5,0.5}
\definecolor{codepurple}{rgb}{0.58,0,0.82}
\definecolor{backcolour}{rgb}{0.99,0.99,0.99}
 
\lstdefinestyle{mystyle}{
    basicstyle=\ttfamily\small,
    backgroundcolor=\color{backcolour},   
    commentstyle=\color{codegreen},
    keywordstyle=\color{magenta},
    numberstyle=\tiny\color{codegray},
    stringstyle=\color{codepurple},
    %basicstyle=\footnotesize,
    breakatwhitespace=false,         
    breaklines=true,                 
    captionpos=b,                    
    keepspaces=false,                 
    numbers=left,                    
    numbersep=10pt,                  
    showspaces=false,                
    showstringspaces=true,
    showtabs=false,                  
    tabsize=2
}
 
\lstset{style=mystyle, label=DescriptiveLabel} 

\let\oldabs\abs
\def\abs{\@ifstar{\oldabs}{\oldabs*}}
\makenomenclature
\begin{document}


%----------------------------------------------------------------------------------------
%   Өөрийн мэдээллээ оруулах хэсэг
%----------------------------------------------------------------------------------------

% Дипломийн ажлын сэдэв
\title{Криптографын зарим алгоритм, программ}
% Дипломын ажлын англи нэр
\titleEng{Some algorithms and programs for cryptography}
% Өөрийн овог нэрийг бүтнээр нь бичнэ
\author{Даянгийн Балжинням}
% Өөрийн овгийн эхний үсэг нэрээ бичнэ
\authorShort{Д. Балжинням}
% Удирдагчийн зэрэг цол овгийн эхний үсэг нэр
\supervisor{Д. Гармаа}
% Хамтарсан удирдагчийн зэрэг цол овгийн эхний үсэг нэр
\cosupervisor{Н. Оюун-Эрдэнэ}

% СиСи дугаар 
\sisiId{20B1NUM0563}
% Их сургуулийн нэр
\university{МОНГОЛ УЛСЫН ИХ СУРГУУЛЬ}
% Бүрэлдэхүүн сургуулийн нэр
\faculty{ХЭРЭГЛЭЭНИЙ ШИНЖЛЭХ УХААН, ИНЖЕНЕРЧЛЭЛИЙН СУРГУУЛЬ}
% Тэнхимийн нэр
\department{МЭДЭЭЛЭЛ, КОМПЬЮТЕРИЙН УХААНЫ ТЭНХИМ}
% Зэргийн нэр
\degreeName{Баклаврын судалгааны ажил}
% Суралцаж буй хөтөлбөрийн нэр
\programeName{Програм хангамж(D061302)}
% Хэвлэгдсэн газар
\cityName{Улаанбаатар}
% Хэвлэгдсэн огноо
\gradyear{2023 оны 11 сар}


%----------------------------------------------------------------------------------------
%   Доорх хэсгийг өөрчлөх шаардлагагүй
%----------------------------------------------------------------------------------------
\include{src/main-pre}

% Удиртгалыг оруулж ирэх ба abstract.tex файлд удиртгалаа бичнэ
\begin{abstract}
	\setcounter{secnumdepth}{0}
	Энэхүү дипломын ажилд криптографийн янз бүрийн алгоритм, программуудыг системтэйгээр судалсан бөгөөд үндсэн зорилго нь тэдгээрийн үндсэн бүтэц, үйл ажиллагааны механизм, практик хэрэглээг ойлгох явдал юм. Энэхүү судалгааны ажилд уламжлалт болон шинээр гарч ирж буй криптографийн алгоритмуудыг судалж, гүйцэтгэл, аюулгүй байдал, үр ашигтай байдалд үндэслэн харьцуулсан судалгааг хийв.\\

	Энэхүү судалгаанд өгөгдлийн шифрлэлтийн стандарт (DES), дэвшилтэт шифрлэлтийн стандарт (AES), Ривест-Шамир-Адлеман (RSA), эллиптик муруй криптографи (ECC) зэрэг тэгш хэмтэй болон тэгш бус криптограф алгоритмуудыг нарийвчлан судалсан.\\

	Төгсөлтийн ажлын практик хэсэгт хэд хэдэн криптографийн програмуудыг боловсруулж, харьцуулсан ба орчин үеийн стандартыг хангасан тоон гарын үсгийн системийг үүлэн технологид суурилан бүтээсэн.
	\section{Зорилго}
	Үүлэн технологид суурилсан тоон гарын үсгийн системийг бүтээснээр хэрэглэгчид өөрсдийн цахим гарын үсгээр баталгаажсан файлуудыг интернет хуваалцах боломжийг бүрдүүлэх гол зорилготой юм.
	\section{Зорилт}
	Бүрэн бүтэн байдал нь хөндөгдөөгүй, эх сурвалж нь тодорхой файлыг хуваалцах боломжийг бүрдүүлэх.
	\section{Үндэслэл}
	Монголд одоогийн байдлаар үүлэн технологид суурилсан тоон гарын үсгийн систем байхгүй байгаа нь хэрэглэгчид энэхүү технологийг ашиглахад төвөгтэй болгож байна.
	

\end{abstract}




%----------------------------------------------------------------------------------------
%   Дипломын үндсэн хэсэг эндээс эхэлнэ
%----------------------------------------------------------------------------------------
%\addcontentsline{toc}{part}{БҮЛГҮҮД}
% Шинэ бүлэг
\chapter{Онолын судалгаа}
\section{Тэгш хэмт крифтограф}
Тэгш хэмт крифтографт шифрлэлт болон шифр тайлах түлхүүрүүд адил байна. Тэгш хэмт алгоритм нь Тэгш бус хэмт шифрлэлтээс харьцангуй хурдан ажилдаг. Гэвч нууцалсан мэдээллийг тайлж унших түлхүүр болон нууцлах түлхүүр адилхан байх нь харилцагч талууд урьдчилан түлхүүрээ хоорондоо тохиролцох шаардлагыг гаргаж ирдэг. Энэ нь сул тал болох эрсдэлтэй. Хэрвээ гуравдагч этгээд түлхүүрийг олж авбал бүх нууцалсан мэдээллийг үзэх боломжтой болох юм.

Хамгийн түгээмэл хэрэглэгддэг тэгш хэмт шифрлэлтийн алгоритм бол Бельгийн криптографич Жоан Даемен, Винсент Рижмен нарын боловсруулсан Advanced Encryption Standard (AES) юм. AES нь хуучин Data Encryption Standard (DES)-ийг сольсон бөгөөд одоо дэлхий даяар ашиглагдаж байна.\cite{AES}
\subsection{Блокон шифрлэлт}

Хэрвээ эх ба шифрлэгдсэн тексүүдийн огторгуй нь ямар нэг $\sum_{}^{n}$ олонлог байвал тухайн криптографыг блокон шифрлэлт гэнэ. Блокон шифрлэлтэд өгсөн мэдээг тэнцүү \textit{n} урттай хэсгүүдэд хуваан шифрлэдэг.\cite{intro_crypo}

Блок шифрт энгийн текстийн блокийг бүхэлд нь авч, шифрлэгдсэн текстийн блокийг үүсгэхэд ашигладаг. Блокийн хэмжээг ерөнхийдөө шифрийн алгоритмаар тодорхойлно. Ихэнх блок шифрүүдийн хувьд энэ нь ихэвчлэн 64 эсвэл 128 бит байдаг ба зарим тохиолдолд нууцлалыг нэмэх зорилгоор 256, 512 бит ч байж болдог.


Хоёр төрлийн алгоритм ашиглах ба нэг нь шифр хийхэд нөгөө нь тайлахад ашиглагддаг. Эдгээр нь \textit{n} урттай бит болон \textit{k} бит урттай түлхүүрийг авч \textit{n} бит урттай блок үүсгэнэ.\\$E: \{0,1\}^k \times \{0,1\}^n \rightarrow \{0,1\}^n$.
	Тайлах алгоритм \textit{D}-г нууцлах функцийн урвуу гэж тодорхойлж болно.\\ $D: \{0,1\}^k \times \{0,1\}^n \rightarrow \{0,1\}^n$\\
$\forall k \in \{0,1\}^k, \forall m \in \{0,1\}^n, D(k, E(k, m)) = m$\\
	\cite{modern_crypto}

	\subsection{Урсгалын шифрлэлт}
	Урсгалын шифрлэлт гэдэг нь өгөгдлийг урсгал маягаар нэг дор нэг битийг Криптографын алгоритм болон түлхүүрээ ашиглан шифрлэх арга юм. Урсгалын шифрийн давуу тал нь блок шифрлэлтээс харьцангуй хурдан ажиллахаас гадна, хэрэгжүүлэлтэд бага код ордог билээ. Гэсэн хэдий ч орчин үед түгээмэл ашиглагдахаа больсон ба элдэв халдагад түгээмэл өртдөг нь үүнтэй холбоотой. Жишээ нь RC4 гэх Урсгалын шифрлэлтийн алгоритм нь WEB болон WPA хамгаалалтад ашиглагддаг байсан хэдий ч хангалттай сайн хамгаалалт болж чадахгүй байгаа тул, хэрэглээнээс халагдаж байна.

	\section{Өгөгдөл шифрлэлтийн стандарт}
	\subsection{DES алгоритм}
	DES (Data Encryption Standard) нь 1970-аад онд хөгжүүлэгдсэн тэгш хэмт блок шифрлэлтийн алгоритм юм. DES нь 64 бит урттай блок дээр ажиллах ба үүнийг 32-бит урттай хоёр хэсэг $L_{0}, R_{0}$ болгон хувааж, баруун талын 32-бит урттай хэсгийг олон янзын аргаар хувиргаж эцэст нь $L_{0}$-тэй XOR үйлдэл хийнэ. Арван зургаан үе хувиргалтын дараагаар $L_{0}, R_{0}$ нийлүүлж 64 бит шифрлэгдсэн блокийг үүсгэнэ.
	\subsubsection{Шинжүүд}
	\begin{enumerate}
		\item Түлхүүрийн урт: DES нь 56 битийн түлхүүрийг ашигладаг бөгөөд анхандаа хангалттай аюулгүй байдлыг хангадаг гэж бодож байсан ч одоо Brute Force халдлагад маш эмзэгт тооцогддог.
		\item Symmetric Encryption: DES нь шифрлэлт болон шифрийг тайлахад ижил түлхүүр ашигладаг. Тиймээс түлхүүрийг илгээгч, хүлээн авагч хоёулаа мэдэж, нууцлах ёстой.

		\item Блок шифр: DES нь тусдаа бит биш харин өгөгдлийн блокууд дээр ажилладаг. Энэ нь их хэмжээний өгөгдлийг шифрлэх шаардлагатай программуудад тохиромжтой.

		\item DES үйлдлүүд: DES нь  Electronic Codebook (ECB), Cipher Block Chaining (CBC), Cipher Feedback (CFB), Output Feedback (OFB), and Counter (CTR) зэрэг хэд хэдэн үйлдлийн горимыг дэмждэг.

		\item DES нь детерминистик: ижил текст болон ижил түлхүүрийн хувьд шифрлэгдсэн текст үргэлж ижил байх болно.
	\end{enumerate}
	хэдийгээр 3-DES гэж байдаг хэдий ч энэ нь тооцоолол ихээр шаарддаг тул цаашид ашиглагдах нь зогссон.

	\subsection{AES}
	АНУ-ын Стандарт, Технологийн үндэсний хүрээлэн (VIST) 1997 онд өгөгдөл нууцлалын стандарт (DES)-ыг сайжруулах ажлыг эхлүүлж 2001 онд В.Рижмень, Д.Дэймен нарын блокон шифрлэлтийн схемийг дэвшилтэт нууцлалын стандартаар зарласан.\cite{intro_crypo}

	AES нь орлуулах сэлгэлт (substitution-permutation) гэж нэрлэгддэг зарчим дээр суурилдаг бөгөөд программ хангамж болон техник хангамжийн аль алин дээр нь хурдан ажилдаг. Орчин үед шифрлэлтийг хурдан хийх зорилгоор техник хангамж дээр зөвхөн энэ алгоритмд зориулсан хэсэг хүртэл байдаг билээ.
	\subsubsection{Үндсэн үйлдэл}
	\begin{enumerate}
		\item \textbf{SubBytes:}
		      \begin{itemize}
			      \item Байт болгоны байрлалыг солино
			      \item Тухайн мөр баганын мэдээлэл солигдоно
		      \end{itemize}
		      \begin{figure}[h]
			      \centering
			      \includegraphics[scale=0.65]{assets/subbytes.png}
			      \caption{SubBytes үйлдэл}
			      \label{fig:subbytes}
		      \end{figure}
		\item \textbf{ShiftRows:}
		      \begin{itemize}
			      \item 1-р мөрийг шилжүүлэхгүй
			      \item 2–р мөрийн байтуудыг зүүн тийш 1 байт шилжүүлнэ
			      \item 3–р мөрийн байтуудыг зүүн тийш 2 байт шилжүүлнэ
			      \item 4–р мөрийн байтуудыг зүүн тийш 3 байт шилжүүлнэ
			      \item Тайлах үйлдлийг хийхдээ баруун тийш шилжүүлэх үйлдлийг хийнэ
		      \end{itemize}
		      \begin{figure}[h]
			      \centering
			      \includegraphics[scale=0.6]{assets/shiftrows.png}
			      \caption{ShiftRows үйлдэл}
			      \label{fig:shiftrows}
		      \end{figure}
		\item \textbf{MixColumns:}
		      \begin{itemize}
			      \item Багана бүр тус тусдаа холигдоно
			      \item Багана болгоны харгалзаа байтууд хоорондоо солигдоно
		      \end{itemize}
		      \begin{figure}[h]
			      \centering
			      \includegraphics[scale=0.6]{assets/mixcolumns.png}
			      \caption{MixColumns үйлдэл}
			      \label{fig:mixcolumns}
		      \end{figure}
		\item \textbf{AddRoundKey:}
		      \begin{itemize}
			      \item 128 бит XOR үйлдлийг циклийн түлхүүрт ашиглана
			      \item Тайлах үйлдэл хийх бол эсрэгээр гүйцэтгэнэ
		      \end{itemize}
		      \begin{figure}[h]
			      \centering
			      \includegraphics[scale=0.6]{assets/addroundkey.png}
			      \caption{AddRoundKey үйлдэл}
			      \label{fig:addroundkey}
		      \end{figure}
	\end{enumerate}

	\subsubsection{AES-ын нууцлалт}
	\begin{enumerate}
		\item шифрлэх блок ба түлхүүрийн урт, мөчлөгийн тоог сонгох. Шифрлэх блок ба түлхүүрийн урт нь 128, 192, 256 байт байж болох бөгөөд мөчлөгийн тоо нь харгалзан 10, 12, 14 байна.
		\item Шифрлэх текст, түлхүүрийн матриц \textit{T, W, K}-г үүсгэнэ.
		\item Эцсийн мөчлөгөөс бусад мөчлөгийн \textit{T, W, K} матрицуудад \textbf{AES}-н үндсэн үйлдлүүдийг дэс дараалан хийнэ. Харин эцсийн мөчлөгт Mix Columns үйлдлийг хийхгүй.
	\end{enumerate}
	% \subsubsection{Нууцын тайлалт}
	% 4x4 for matrix below
	\begin{center}
		$\begin{bmatrix}
				b_{0} & b_{4} & b_{8}  & b_{12} \\
				b_{1} & b_{5} & b_{9}  & b_{13} \\
				b_{2} & b_{6} & b_{10} & b_{14} \\
				b_{3} & b_{7} & b_{11} & b_{15} \\
			\end{bmatrix}$
	\end{center}
	\subsection{РСА (RSA)}
	РСА (RSA) нь анхны тооны өвөрмөц шинж чанарыг ашигладаг тэгш бус хэмтэй шифрлэлтийн арга юм. Анх 1977 онд танилцуулагдсан ба, өнөөг хүртэл хэрэглээнд хэвээр байгаа. Өнөөдрийн дэлхий даяар мөрдөгдөж байгаа стандарт нь хоёр анхны тооны үржвэр болох модулус нь 2048 бит хэмжээтэй байх ёстой. Энэ нь 617 оронтой тоо байна гэсэн үг юм.
	\begin{itemize}
		\item Хоёр анхны тоо болох $p$ болон $q$ сонгоно.
		\item $n = p*q$ утгыг олно.
		\item $\phi(n) = (p-1)*(q-1)$ утгыг олно.
		\item Дараах нөхцөлийг хангах $e$ тоог сонгоно $1 < e < \phi(n)$ ба хиех$(e, \phi(n)) = 1$.
		\item $d$ нь $d \equiv e^{-1} \mod \phi(n)$ гэж тодорхойлогдоно.
	\end{itemize}

	Нийтийн түлхүүр нь $(e, n)$ болох ба хувийн түлхүүр нь $(d, n)$ болно.\cite{РСА (RSA)}
	
\subsubsection{Нууцлал}
РСА (RSA) алгоритмын нууцлал маш том хэмжээний анхны тоог хоёр тооны үржигдэхүүн болгон задлах дээр тогтдог ба өнөөгийн бидний машины тооцон бодох чадал хараахан хангалттай биш байгаа юм.

\begin{table}[h!]
	\centering
	\caption{Муйхар хүчний алгоритм ашиглан РСА (RSA) нууцлалыг эвдэх нь \cite{Brute-force-РСА (RSA)}}
	\begin{tabular}{|c|c|c|c|}
	\hline
	n & p*q & Оролдого (Хайлт) & Хугацаа (секунд) \\
	\hline
	187 & $11 \times 17$ & 5 & $0.00344800949097$ \\
	913 & $11 \times 83$ & 5 & $0.00358390808105$ \\
	14041 & $19 \times 739$ & 8 & $0.004469871521$ \\
	557009 & $653 \times 853$ & 119 & $0.00167393684387$ \\
	9192907 & $937 \times 9811$ & 159 & $0.00201606750488$ \\
	37675201 & $3907 \times 9643$ & 540 & $0.0139532089233$ \\
	17614895377 & $40559 \times 434303$ & 4252 & $0.117401838303$ \\
	599855115407 & $694789 \times 863363$ & 56166 & $0.712327957153$ \\
	4684589242027 & $837533 \times 5593319$ & 66714 & $1.99000310898$ \\
	6833740248499 & $2565161 \times 2664059$ & 187492 & $2.51408982277$ \\
	91063247464523 & $9577907 \times 35324489$ & 188371 & $8.69724798203$ \\
	\hline
	\end{tabular}
	\end{table}

	Хамгийн сүүлд үржигдэхүүнд задалж чадсан буюу нууцлал нь амжилттай эвдэгдсэн нь РСА (RSA)-250 буюу 829 бит урттай байгаа юм. Фабрис Будот, Пьеррик Гаудри, Ауроре Гилевич, Надия Хенингер, Эммануэль Томе, Пол Циммерманн нараар ахлуулсан судлаачдын баг үүнийг 2020 онд гүйцэтгэсэн. Тооцоололд ойролцоогоор 2700 цөм жил \footnote{Цөм жил гэдэг нь CPU-ний нэг цөмийг бүтэн жил ашигласантай тэнцэнэ.} зарцуулагдсан бөгөөд шигших үе шат нь хуанлийн 35 долоо хоног янз бүрийн машинууд дээр хийгдсэн.

	\begin{figure}[h]
		\centering
		\includegraphics[scale=0.73]{assets/rsacomplexity.png}
		\caption{Хугацааны ээдрээ}
		\label{fig:rsacomplexity}
	\end{figure}

РСА (RSA) үржигдэхүүн задлах нь(factoring) цифрийн тоо нэмэгдэх тусам илтгэгч функцээр хугацааны ээдрээ тооцогдох тул одоогийн байдлаар РСА (RSA) 1024, РСА (RSA) 2048 нь хангалттай аюулгүй байгаа бөгөөд дэлхий нийтээрээ ашиглаж байна. Энэ нь дээрх диаграммаас харагдана.

\subsection{ECC (Эллипс муруйлаг криптограф)}

Эллипс муруйлаг криптографи (ECC) нь хязгаарлагдмал талбар дээрх эллипс муруйнуудын алгебрийн бүтцэд суурилсан нийтийн түлхүүрийн криптографийн нэг төрөл юм. Том бүхэл тоонуудын үржвэр дээр суурилдаг RSA-аас ялгаатай нь ECC нь эллиптик муруй дискрет логарифмын бодлогыг (ECDLP) шийдвэрлэхэд хүндрэлтэй байдгаас аюулгүй байдлаа олж авдаг. RSA-аас ECC-ийн мэдэгдэхүйц давуу тал нь түүний үр ашигтай байдал юм; ECC нь RSA-тай ижил түвшний аюулгүй байдлыг RSA-н хажууд асар жижиг хэмжээтэй түлхүүрээр олгодог. Үр ашиг нь илүү хурдан тооцоолол, эрчим хүчний бага зарцуулалт, илүү жижиг хэмжээтэй түлхүүр гэх мэт орох ба ECC нь хөдөлгөөнт төхөөрөмж, ухаалаг карт зэрэг хязгаарлагдмал нөөцтэй төхөөрөмжүүдэд илүү тохиромжтой.
\subsection{ECC ба RSA харьцуулалт}
\begin{table}[h]
	\centering
	\begin{tabular}{|c|c|c|}
	\hline
	\textbf{Битийн аюулгүй байдлын түвшин} & \textbf{RSA бит хэмжээ} & \textbf{ECC бит хэмжээ} \\ \hline
	80                          & 1024                      & 160                       \\ \hline
	112                         & 2048                      & 224                       \\ \hline
	128                         & 3072                      & 256                       \\ \hline
	192                         & 7680                      & 384                       \\ \hline
	256                         & 15360                     & 512                       \\ \hline
	\end{tabular}
	\caption{Аюулгүй байдлын түвшин ба RSA болон ECC түлхүүрийн хэмжээг харьцуулах \cite{RSAvsECC}}
	\label{tab:rsa_ecc_key_sizes}
	\end{table}
	
	\begin{figure}[h]
		\centering
		\includegraphics[scale=0.6]{assets/graphs/keysize.png}
		\caption{RSA ба ECC түлхүүрийн хэмжээнүүдийн харьцуулалт}
		\label{fig:architecture}
	\end{figure}
	\begin{figure}[h]
		\centering
		\includegraphics[scale=0.65]{assets/graphs/1.png}
		\caption{8 бит өгөгдөл шифрлэлт}
		\label{fig:architecture}
	\end{figure}
	\begin{figure}[h]
		\centering
		\includegraphics[scale=0.65]{assets/graphs/2.png}
		\caption{64 бит өгөгдөл шифрлэлт}
		\label{fig:architecture}
	\end{figure}
	\begin{figure}[h]
		\centering
		\includegraphics[scale=0.65]{assets/graphs/3.png}
		\caption{256 бит өгөгдөл шифрлэлт}
		\label{fig:architecture}
	\end{figure}
	\begin{figure}[h]
		\centering
		\includegraphics[scale=0.65]{assets/graphs/4.png}
		\caption{8 бит өгөгдөл шифрлэлт тайлалт}
		\label{fig:architecture}
	\end{figure}
	\begin{figure}[h]
		\centering
		\includegraphics[scale=0.65]{assets/graphs/5.png}
		\caption{64 бит өгөгдөл шифрлэлт тайлалт}
		\label{fig:architecture}
	\end{figure}
	\begin{figure}[h]
		\centering
		\includegraphics[scale=0.65]{assets/graphs/6.png}
		\caption{256 бит өгөгдөл шифрлэлт тайлалт}
		\label{fig:architecture}
	\end{figure}
	\begin{figure}[h]
		\centering
		\includegraphics[scale=0.65]{assets/graphs/7.png}
		\caption{8 бит өгөгдөл хугацааны харьцуулалт}
		\label{fig:architecture}
	\end{figure}
	\begin{figure}[h]
		\centering
		\includegraphics[scale=0.65]{assets/graphs/8.png}
		\caption{64 бит өгөгдөл хугацааны харьцуулалт}
		\label{fig:architecture}
	\end{figure}
	\begin{figure}[h]
		\centering
		\includegraphics[scale=0.65]{assets/graphs/9.png}
		\caption{256 бит өгөгдөл хугацааны харьцуулалт}
		\label{fig:architecture}
	\end{figure}
	
	
	\begin{table}
	\centering
	\caption{8 бит өгөгдөл – шифрлэлт ба шифр тайлах хугацаа (Секундээр) \cite{RSAvsECC}}
	\begin{tabular}{|c|c|c|c|c|c|c|}
	\hline
	Хамгаалалт & ECC шифр & RSA шифр & ECC тайлах & RSA тайлах & ECC & RSA \\
	\hline
	80 & 0.4885 & 0.0307 & 1.3267 & 0.7543 & 1.8152 & 0.7850 \\
	112 & 2.2030 & 0.0299 & 1.5863 & 2.7075 & 3.7893 & 2.7375 \\
	128 & 3.8763 & 0.0305 & 1.7690 & 6.9409 & 5.6453 & 6.9714 \\
	144 & 4.7266 & 0.0489 & 2.0022 & 13.6472 & 6.7288 & 13.6962 \\
	\hline
	\end{tabular}
	\end{table}
	
	\begin{table}
	\centering
	\caption{64 бит өгөгдөл – шифрлэлт ба шифр тайлах хугацаа (Секундээр) \cite{RSAvsECC}}
	\begin{tabular}{|c|c|c|c|c|c|c|}
	\hline
	Хамгаалалт & ECC шифр & RSA шифр & ECC тайлах & RSA тайлах & ECC  & RSA  \\
	\hline
	80 & 2.1685 & 0.1366 & 5.9099 & 5.5372 & 8.0784 & 5.6738 \\
	112 & 9.9855 & 0.1635 & 6.9333 & 20.4108 & 16.9188 & 20.5743 \\
	128 & 15.0882 & 0.1672 & 7.3584 & 46.4782 & 22.4466 & 46.6454 \\
	144 & 20.2308 & 0.1385 & 8.4785 & 77.7642 & 28.7093 & 77.9027 \\
	\hline
	\end{tabular}
	\end{table}
	
	\begin{table}
	\centering
	\caption{256 бит өгөгдөл – шифрлэлт ба шифр тайлах хугацаа (Секундээр) \cite{RSAvsECC}}
	\begin{tabular}{|c|c|c|c|c|c|c|}
	\hline
	Хамгаалалт & ECC шифр & RSA шифр & ECC тайлах & RSA тайлах & ECC  & RSA  \\
	\hline
	80 & 7.9240 & 0.5596 & 22.8851 & 19.3177 & 30.8091 & 19.8772 \\
	112 & 39.7008 & 0.5815 & 26.3331 & 102.0337 & 66.0339 & 102.6153 \\
	128 & 58.4386 & 0.5611 & 27.4060 & 209.6086 & 85.8446 & 210.1697 \\
	144 & 77.5034 & 0.5718 & 32.1522 & 311.0649 & 109.6556 & 311.6368 \\
	\hline
	\end{tabular}
	\end{table}
	
\chapter{Системийн зохиомж}
\section{Тоон гарын үсгийн стандарт}
Хэдийгээр бүх цахим гарын үсэг нь DSS-ийн дүрмийг дагаж мөрдөх ёстой боловч тэдгээр нь бүгд адилхан биш юм. Баримт бичигт гарын үсэг зурахад ашиглаж болох гурван төрлийн тоон гарын үсгийн стандарт байдаг.
\begin{enumerate}
	\item \textbf{Энгийн цахим гарын үсэг (SES)} - Цахим гарын үсгийн хамгийн үндсэн хэлбэр. SES нь баримт бичигт нэмэхэд хурдан бөгөөд хялбар боловч шифрлэлтийн аргаар хамгаалагдаагүй. Өөрөөр хэлбэл, тийм ч аюулгүй биш юм. Үүнд жишээ нь цахим шуудангийн гарын үсэг ордог.
	\item \textbf{Нарийвчилсан цахим гарын үсэг (AES)} - Хэдийгээр хууль ёсны дагуу хүчингүй боловч AES (Advanced Electronic Signature) нь гарын үсэг зурсны дараа баримт бичигт өөрлчлөлт орсон эсэхийг мэдэх боломжтой крифтографыг ашигладаг. Гэсэн хэдий ч хуулийн дагуу хүчингүй хэвээр.
	\item \textbf{Qualified advanced electronic signature (QES)} - Цахим хэлбэрээр гарын үсэг зурах хамгийн найдвартай арга. Тоон гарын үсэг гэж нэрлэгддэг шаардлага хангасан цахим гарын үсэг нь аюулгүй байдлын дээд түвшинг хангахын тулд нийтийн түлхүүрийн дэд бүтэц, тэгш бус криптограф, Two Factor баталгаажуулалтыг ашигладаг. Эдгээрийг ашигласнаар, гарын үсэг нь хууль ёсны дагуу хүчийн төгөлдөр болно.
\end{enumerate}
\section{Адил системийн судалгаа}
\subsubsection{Tridumkey.mn}
Tridimkey нь Монгол улсын бүртгэлийн ерөнхий газраар хүлээн зөвшөөрөгдсөн тоон гарын үсэг олгогч ба байгуулагад зориулж гарын үсэг олгодог нь онцлог санагдсан. Байгууллагад зориулж гарын үсэг авахад бүрдүүлдэг баримтууд.
\begin{enumerate}
	\item Иргэний үнэмлэх эх хувь эсвэл И-монголиа-ийн иргэний  үнэмлэхийн лавлагаа
	\item Байгууллагын гэрчилгээ
	\item Албан бичиг эх хувь		Загвар татах
	\item Эзэмшигч өөрийн биеээр ирэх боломжгүй үед итгэмжлэлтэй албан бичиг
	\item Анкет
\end{enumerate}
Гэвч сул тал нь энэхүү тоон гарын үсгийн систем нь зөвхөн \textbf{Windows} үйлдлийн систем дээр ажилдаг ба Macos эсвэл Linux үйлдлийн систем ашигладаг хэрэглэгчид ашиглах боломжгүй болж байгаа юм.
\subsubsection{Monpass.mn}
"Таньж баталгаажуулах тоон гарын үсгийн гэрчилгээ: Цахим бизнес, төрийн болон бусад төрөл бүрийн систем, онлайн үйлчилгээнд хандах, бусад цахим гүйлгээ, хэлцэл хийхэд найдвартай таньж баталгаажуулах, захидал харилцааг хөдөлбөргүй баталгаажуулахын тулд тоон гарын үсэг зурах, захидал харилцаа, дамжуулж буй баримт бичгийг шифрлэн дамжуулах, ажилтнууд, хэрэглэгчдийг хялбар таних, бөөний онлайн худалдаа зохион байгуулах гэх мэт зорилгоор ашиглагддаг тоон гарын үсгийн гэрчилгээ – цахим баримт бичиг юм. Энэ гэрчилгээ нь хэрэглэгчийн мэдээлэл, олгосон ГОБ-ын мэдээлэл, хосгүй серийн дугаар болон бусад хосгүй өгөгдлүүд, хүчинтэй хугацаа, тоон гарын үсгийн нийтийн түлхүүр, холбогдох бусад мэдээллийг агуулсан байх бөгөөд Хувь хүмүүс болон байгууллагын төлөөлөгч хэн боловч ашиглаж болно. Захидал, мэдээлэлдээ тоон гарын үсэг зурахдаа өөрийн тоон гарын үсгийн хувийн түлхүүрийг ашиглах ба харин шифрлэн илгээх бол хүлээн авагчийн нийтийн түлхүүрийг ашиглана." гэсэн танилцуулагатай байсан ба гүнзгий судалж үзэхэд мөн л хэрэглэгчийн үйлдлийн систем зөвхөн \textbf{Windows} байж л тоон гарын үсгийн ашиглах боломжтой байсан юм.
\pagebreak
\section{Системийн шаардлага}
\subsubsection{Функциональ шаардлагуудыг дараах хүснэгтэд тодорхойлов}
% table
\begin{table}[h]
	\centering
	\caption{Функциональ шаардлага}
	\begin{tabular}{ |p{2cm}|p{13cm}| }
		\hline
		ФШ 100 & Систем нь хэрэглэгчийн тоон гарын үсэг үүсгэх чадвартай байх ёстой. Үүнд хэрэглэгч бүрийн өвөрмөц түлхүүрийн хослолыг бий болгох орно.                                                                             \\ \hline
		ФШ 200 & Систем нь тоон гарын үсгийг баталгаажуулах функцээр хангах ёстой. Энэ нь гарын үсэг зурсан баримт бичгийг хүлээн авч, гарын үсэг зурсан хүний нийтийн түлхүүрийг ашиглан гарын үсгийг баталгаажуулах ёстой.        \\ \hline
		ФШ 300 & Систем нь хэрэглэгчдэд гарын үсэг зурахын тулд янз бүрийн форматтай цахим баримт бичгүүдийг (жишээлбэл, .doc, .pdf, .xls гэх мэт) байршуулахыг зөвшөөрөх ёстой.                                                    \\ \hline
		ФШ 400 & Систем нь хэрэглэгчдийг баримт бичигт гарын үсэг зурах, баталгаажуулахаас өмнө баталгаажуулах ёстой. Үүнийг хэрэглэгчийн нэр/нууц үг, олон хүчин зүйлийн баталгаажуулалт эсвэл бусад аюулгүй аргуудаар хийж болно. \\ \hline
		ФШ 500 & Систем нь баримт бичиг байршуулах, гарын үсэг үүсгэх, гарын үсгийн баталгаажуулалт зэрэг хэрэглэгчдийн хийсэн бүх үйлдлийг бүртгэх ёстой.                                                                          \\ \hline
		ФШ 600 & Систем нь бусад үйлчилгээтэй нэгтгэх API-г өгөх ёстой. Энэ нь бусад програм хангамж эсвэл үйлчилгээнд энэ үйлчилгээний тоон гарын үсгийн чадварыг ашиглах боломжийг олгоно.                                        \\  \hline
		ФШ 700 & Веб нь хэрэглэгч бүртгэх боломжтой байх                                                                                                                                                                            \\ \hline
	\end{tabular}
\end{table}
\pagebreak
\subsubsection{Функциональ бус шаардлагуудыг дараах хүснэгтэд тодорхойлов}
% table
\begin{table}[h!]
	\centering
	\caption{Функциональ бус шаардлага}
	\begin{tabular}{ |p{2cm}|p{13cm}| }
		\hline
		ФБШ 100 & Систем нь GDPR эсвэл HIPAA гэх мэт холбогдох бүх мэдээллийн аюулгүй байдал, нууцлалын дүрэм журмыг дагаж мөрдөх ёстой. Гарын үсэг, баримт бичиг зэрэг бүх өгөгдөл шифрлэгдсэн байх ёстой. \\ \hline
		ФБШ 200 & Систем нь гүйцэтгэлийн бууралтгүйгээр олон тооны хэрэглэгчид болон баримт бичгүүдийг зохицуулах чадвартай байх ёстой.                                                                     \\ \hline
		ФБШ 300 & Үүлэн үйлчилгээ нь хамгийн бага зогсолттой, 24/7 цагийн турш ашиглах боломжтой байх ёстой. Үйлчилгээний түвшний гэрээ (SLA) нь дор хаяж 99.9\% ажиллах хугацааг баталгаажуулах ёстой.     \\ \hline
		ФБШ 400 & Систем нь хүлээн зөвшөөрөгдсөн тодорхой хугацааны дотор гарын үсэг үүсгэх, баталгаажуулах хүсэлтийг хурдан боловсруулах чадвартай байх ёстой.                                             \\ \hline
		ФБШ 500 & Систем нь янз бүрийн техникийн чадвартай хэрэглэгчдэд үүнийг үр дүнтэй ашиглах боломжийг олгодог хэрэглэгчдэд ээлтэй интерфэйстэй байх ёстой.                                             \\ \hline
		ФБШ 600 & Үүлэн үйлчилгээ нь янз бүрийн үйлдлийн систем, хөтөч, төхөөрөмжтэй нийцтэй байх ёстой.                                                                                                    \\  \hline
		ФБШ 700 & Энэ систем нь гамшгийн үед өгөгдөл алдагдахгүй байхын тулд найдвартай нөөцлөх, сэргээх механизмтай байх ёстой.                                                                            \\ \hline
		ФБШ 800 & Систем нь Европ дахь eIDAS эсвэл АНУ-ын ESIGN хууль зэрэг тоон гарын үсгийн хууль тогтоомж, дүрэм журамд нийцсэн байх ёстой.                                                              \\ \hline
	\end{tabular}
\end{table}
\newpage
\section{Use case диаграм}
\begin{figure}[h]
	\centering
	\includegraphics[scale=0.66]{assets/usecase_mn.png}
	\caption{Use case диаграм}
	\label{fig:usecasemn}
\end{figure}
\pagebreak
\section{Sequence диаграм}
\begin{figure}[h!]
	\centering
	\includegraphics[scale=0.45, angle=90]{assets/sequence2.png}
	\caption{Sequence диаграм}
	\label{fig:usecasemn}
\end{figure}
\newpage
\section{Өгөгдлийн сангийн диаграм}

\begin{figure}[h!]
	\centering
	\includegraphics[scale=0.46]{assets/cryptography.png}
	\caption{Датабаз диаграм}
	\label{fig:dbdiagram}
\end{figure}
\break
\newpage
\section{Архитектур}
Энэхүү төслийг ажиллахад илүү хямд зардалтай хүртээмжтэй, ачаалал даах чадварыг нэмэх зорилгоор серверлесс Архитектур сонгосон юм. Фронт-энд хэсэг нь NextJS-н ашигласан тул Сервер талын рендер хийж байгаа ба Бак-энд хэсэг нь тэр чигтээ AWS-н Ламдба функц дээр ажиллах юм. Хэрэглэгчийн серверлүү илгээж байгаа бичиг баримтыг AWS-н Ламдба дээр үүсгэсэн нэг удаагийн холбоосоор хэрэглэгч шууд AWS-руу оруулах юм. Өмнө нь Whatsapp ийм маягаар файл оруулдаг байсан жишээнээс санаа авсан. Статик файлуудыг AWS-н Cloud Front дээр байрлуулсан ба энэ нь дэлхийн өнцөг бүрт байдаг хэрэглэгчид хамгийн ойрхан контент түгээх сүлжээ юм энэ нь хэрэглэгчид илүү хурдан татах боломжийг олгохоос гадна мөн сагхүүгийн хувьд хэмнэлттэй болдог юм. Хэт их хэмжээний хандалт, халдлага зэргээс сэргийлэх зорилгоор AWS-WAF ашиглаж бүх хүсэлтүүд илгээгдэнэ.
\begin{figure}[h!]
	\centering
	\includegraphics[scale=0.32]{assets/server.png}
	\caption{Архитектур}
	\label{fig:architecture}
\end{figure}
\pagebreak
\newpage
\section{Өгөгдлийн сангийн хүснэгтүүд}
\section{ER диаграм}
\begin{table}[h]
	\caption{User хүснэгт}
	\begin{tabular}{|l|l|l|p{8cm}|}
	\hline
	№ &  Талбарын нэр & Өгөгдлийн төрөл & Тайлбар \\ \hline
	1 &  id & Varchar & Хэрэглэгчийн дахин давтагдашгүй ID-г хад-
	гална\\ \hline
	2 &  email & Varchar & Хэрэглэгчийн цахим шууданг хадгална\\ \hline
	3 &  password & Varchar & Хэрэглэгчийн нууц үгийг шифрлэж, энэ талбарт хадгална \\ \hline
    4 &  name & Varchar & Хэрэглэгчийн интерфейсээс оруулсан хэрэглэгчийн нэр. Зөвхөн латин үсгийг хадгална. \\ \hline
	5 &  emailVerified & DateTime & Хэрэглэгчийн имэйлийг баталгаажуулсан цагийн тэмдэ \\ \hline
	6 &  image & Varchar & Хэрэглэгчийн байршуулсан зургийн холбоос хадгалагдах бөгөөд зам нь энэ талбарт хадгалагдана \\ \hline
	7 &  role & ENUM & Хэрэглэгчийн нэвтрэлтийн эрх (USER, ADMIN) \\ \hline
\end{tabular}
\end{table}


\begin{table}[h]
	\caption{Session хүснэгт}
	\begin{tabular}{|l|l|l|p{8cm}|}
	\hline
	№ &  Талбарын нэр & Өгөгдлийн төрөл & Тайлбар \\ \hline
	1 &  id & Varchar & Нэвтрэлтийн түүхийн өвөрмөц ID \\ \hline
	2 &  sessionToken & Varchar & Гуравдагч этгээдийн токен (Github, Google) байна. \\ \hline
	3 &  userId & Varchar & Хэрэглэгчийн өвөрмөц ID \\ \hline
	4 &  expires & DateTime & Дуусах хугацаа \\ \hline
\end{tabular}
\end{table}

\begin{table}[h]
	\caption{VerificationToken хүснэгт}
	\begin{tabular}{|l|l|l|p{8cm}|}
	\hline
	№ &  Талбарын нэр & Өгөгдлийн төрөл & Тайлбар \\ \hline
	1 &  identifier & Varchar & Токенд зориулсан өвөрмөц ID\\ \hline
	2 &  token & Varchar & Баталгаажуулалтын Токен \\ \hline
	3 &  expires & DateTime & Дуусах хугацаа \\ \hline
\end{tabular}
\end{table}

\begin{table}[h]
	\caption{Account хүснэгт}
	\begin{tabular}{|l|l|l|p{8cm}|}
	\hline
	№ &  Талбарын нэр & Өгөгдлийн төрөл & Тайлбар \\ \hline
	1 &  id & Varchar & Өвөрмрц ID \\ \hline
	2 &  userId & Varchar & Энэ бүртгэлтэй холбоотой хэрэглэгчийн ID \\ \hline
	3 &  type & Varchar & Бүртгэлийн тө \\ \hline
	4 &  provider & Varchar & Аль гуравдагч этгээдийг дамжиж нэвтэрсэн (Github, Google) \\ \hline
	5 &  providerAccountId & Varchar & Хаягийн өвөрмөц ID \\ \hline
    6 &  refresh\_token & Varchar & Шинэ токен үүсгэх нууц үг\\ \hline
    7 &  access\_token & Varchar & Баталгаажуулах токен \\ \hline
    8 &  expires\_at & Int & Дуусах хугацаа\\ \hline
    9 &  token\_type & Varchar & Төрөл \\ \hline
    10 &  scope & Varchar & Нэвтрэлтийн эрх \\ \hline
    11 &  id\_token & Varchar & Өвөрмөц ID \\ \hline
    12 &  session\_state & Varchar & Одоо нэвтрэлттэй байгаа эсэх\\ \hline
\end{tabular}
\end{table}

\begin{table}[h]
	\caption{UserUploadedFiles хүснэгт}
	\begin{tabular}{|l|l|l|p{8cm}|}
	\hline
	№ &  Талбарын нэр & Өгөгдлийн төрөл & Тайлбар \\ \hline
	1 &  id & Varchar & Хэрэгдэгчийн оруулсан файлын өвөрмөц ID\\ \hline
	2 &  userId & Varchar & Файлыг байршуулсан хэрэглэгчийн ID \\ \hline
	3 &  fileName & Varchar & Файлын нэр\\ \hline
	4 &  filePath & Varchar & Файл хадгалагдаж буй зам \\ \hline
	5 &  createdAt & DateTime & Файлыг байршуулсан цаг \\ \hline
	6 &  updatedAt & DateTime & Файлын мэдээлэл хамгийн сүүлд шинэчлэгдсэн цаг \\ \hline
\end{tabular}
\end{table}

\begin{table}[h]
	\caption{OtpSecret хүснэгт}
	\begin{tabular}{|l|l|l|p{8cm}|}
	\hline
	№ &  Талбарын нэр & Өгөгдлийн төрөл & Тайлбар \\ \hline
	1 &  id & Varchar & Нэг удаагийн нууц үг үүсгэх түлхүүрийн ID\\ \hline
	2 &  userId & Varchar & Холбоотой хэрэглэгчийн ID \\ \hline
	3 &  isVerified & Boolean & OTP нь баталгаажсан эсэх \\ \hline
	4 &  secret & Text & Нэг удаагийн нууц үгийн баталгаажуулалтад ашигласан нууц \\ \hline
	5 &  createdAt & DateTime & OTP үүсгэсэн цаг\\ \hline
	6 &  updatedAt & DateTime & Хамгийн сүүлд шинэчилсэн цаг\\ \hline
\end{tabular}
\end{table}

\begin{table}[h]
	\caption{SignatureDigest хүснэгт}
	\begin{tabular}{|l|l|l|p{8cm}|}
	\hline
	№ &  Талбарын нэр & Өгөгдлийн төрөл & Тайлбар \\ \hline
	1 &  id & Varchar & Нууц үгийн хайшийн ID\\ \hline
	2 &  fileName & Varchar & Холбогдсон файлын нэр \\ \hline
	3 &  userId & Varchar & Харгалзах хэрэглэгчийн ID\\ \hline
	4 &  digest & Text & Хайшын утга \\ \hline
	5 &  createdAt & DateTime & Үүсгэсэн огноо \\ \hline
	6 &  updatedAt & DateTime & Шинэчилсэн огноо \\ \hline
\end{tabular}
\end{table}

\begin{table}[h]
	\caption{UserGeneratedKeys хүснэгт}
	\begin{tabular}{|l|l|l|p{8cm}|}
	\hline
	№ &  Талбарын нэр & Өгөгдлийн төрөл & Тайлбар \\ \hline
	1 &  id & Varchar & Өвөрмөц ID\\ \hline
	2 &  userId & Varchar & Түлхүүрийг үүсгэсэн хэрэглэгчийн ID \\ \hline
	3 &  publicKeyLink & Varchar & Нийтийн түлхүүрийн байршил \\ \hline
	4 &  privateKeyLink & Varchar & Хувийн түлхүүрийн байршил \\ \hline
	5 &  createdAt & DateTime & Үүсгэсэн огноо \\ \hline
	6 &  updatedAt & DateTime & Шинэчилсэн огноо \\ \hline
\end{tabular}
\end{table}
\pagebreak
\newpage





\chapter{Тайлан боловсруулах зөвлөмж}
\subfile{writing.tex}

\chapter{Бичвэр боловсруулалт}
Бүлгийн гарчгийн дор тухайн бүлэгт юу агуулж байгаа, юуны талаар өгүүлэхийг товч бичих нь баримт бичгийг уншигчдад илүү ойлгомжтой болгодог.
% Бүлгийн дэд гарчиг
\section{Шинэ мөр ба цогцолбор}
Латекс бичих явцад олон хоосон зай, шинэ мөр авахад гаралтын файлд ганцхан хоосон зайгаар дүрсэлж харуулдгаараа бусад засварлагчаас ялгаатай юм.

Шинэ мөр буюу цогцолбор (paragraph) авахдаа хоёр удаа enter товч дарах буюу нэг хоосон мөр үлдээж бичнэ. \par Эсвэл par командыг бичнэ.
\\Харин шинэ мөр авахдаа хоёр ширхэг ургашаа налуу зураас дарааллуулан бичнэ.  Дэлгэрэнгүйг \cite{pharagraph1}-с унш.

Contrary to popular belief, Lorem Ipsum is not simply random text. It has roots in a piece of classical Latin literature from 45 BC, making it over 2000 years old. Richard McClintock, a Latin professor at Hampden-Sydney College in Virginia, looked up one of the more obscure Latin words, consectetur, from a Lorem Ipsum passage, and going through the cites of the word in classical literature, discovered the undoubtable source. Lorem Ipsum comes from sections 1.10.32 and 1.10.33 of "de Finibus Bonorum et Malorum" (The Extremes of Good and Evil) by Cicero, written in 45 BC. This book is a treatise on the theory of ethics, very popular during the Renaissance. The first line of Lorem Ipsum, "Lorem ipsum dolor sit amet..", comes from a line in section 1.10.32.

The standard chunk of Lorem Ipsum used since the 1500s is reproduced below for those interested. Sections 1.10.32 and 1.10.33 from "de Finibus Bonorum et Malorum" by Cicero are also reproduced in their exact original form, accompanied by English versions from the 1914 translation by H. Rackham.

\section{Бичвэр зэрэгцүүлэх}
\subsection{Зүүн тийш зэрэгцүүлэх}
\begin{flushleft}
	Contrary to popular belief, Lorem Ipsum is not simply random text. It has roots in a piece of classical Latin literature from 45 BC, making it over 2000 years old. Richard McClintock, a Latin professor at Hampden-Sydney College in Virginia, looked up one of the more obscure Latin words, consectetur, from a Lorem Ipsum passage, and going through the cites of the word in classical literature, discovered the undoubtable source. Lorem Ipsum comes from sections 1.10.32 and 1.10.33 of "de Finibus Bonorum et Malorum" (The Extremes of Good and Evil) by Cicero, written in 45 BC. This book is a treatise on the theory of ethics, very popular during the Renaissance. The first line of Lorem Ipsum, "Lorem ipsum dolor sit amet..", comes from a line in section 1.10.32.

	The standard chunk of Lorem Ipsum used since the 1500s is reproduced below for those interested. Sections 1.10.32 and 1.10.33 from "de Finibus Bonorum et Malorum" by Cicero are also reproduced in their exact original form, accompanied by English versions from the 1914 translation by H. Rackham.
\end{flushleft}

\subsection{Баруун тийш зэрэгцүүлэх}
\begin{flushright}
	Contrary to popular belief, Lorem Ipsum is not simply random text. It has roots in a piece of classical Latin literature from 45 BC, making it over 2000 years old. Richard McClintock, a Latin professor at Hampden-Sydney College in Virginia, looked up one of the more obscure Latin words, consectetur, from a Lorem Ipsum passage, and going through the cites of the word in classical literature, discovered the undoubtable source. Lorem Ipsum comes from sections 1.10.32 and 1.10.33 of "de Finibus Bonorum et Malorum" (The Extremes of Good and Evil) by Cicero, written in 45 BC. This book is a treatise on the theory of ethics, very popular during the Renaissance. The first line of Lorem Ipsum, "Lorem ipsum dolor sit amet..", comes from a line in section 1.10.32.

	The standard chunk of Lorem Ipsum used since the 1500s is reproduced below for those interested. Sections 1.10.32 and 1.10.33 from "de Finibus Bonorum et Malorum" by Cicero are also reproduced in their exact original form, accompanied by English versions from the 1914 translation by H. Rackham.
\end{flushright}


\section{Хэлбэржилт}
Энэ бүлэгт бичвэрийг хэлбэржүүлэх (format) командуудын талаар дурьдана. Илүү дэлгэрэнгүйг \cite{format1}-с хар.

\subsection{Тодруулах}
\texttt{\textbackslash textbf} командаар бичвэрийг \textbf{тодруулах буюу болд} болгоно.

\subsection{Налуулах}
\texttt{\textbackslash textit} командаар бичвэрийг \textit{бичмэл буюу италик} болгоно.

\subsection{Доогуур зураас}
\texttt{\textbackslash underline} командаар бичвэрийг \textbf{тодруулах буюу болд} болгоно.

\section{URL оруулах}
\texttt{\textbackslash url} команд дотор холбоосыг бичнэ. \url{http://milab.num.edu.mn}


\section{Жагсаалт}
\subsection{Энгийн жагсаалт}
\texttt{\textbackslash begin\{itemize\}} командын дотор энгийн жагсаалтыг бичнэ \cite{list}.
\begin{itemize}
	\item Жагсаалтын эхний элемент
	\item Жагсаалтын хоёрдугаар элемент
	\item Жагсаалтын гуравдугаар элемент
	\item Жагсаалтын дөрөвдүгээр элемент
\end{itemize}

\subsection{Дугаартай жагсаалт}
\texttt{\textbackslash begin\{enumerate\}} командын дотор энгийн жагсаалтыг бичнэ \cite{list}.
\begin{enumerate}
	\item Жагсаалтын эхний элемент
	\item Жагсаалтын хоёрдугаар элемент
	\item Жагсаалтын гуравдугаар элемент
	\item Жагсаалтын дөрөвдүгээр элемент
\end{enumerate}


\chapter{Ишлэл, зүүлт}
\section{Ишлэл}
Ашигласан материал эсвэл номзүйг бичвэр тодор ишлэхдээ cite командаар заалтыг нь оруулна.
Үүний тулд энэ хуудасны хамгийн доор байгаа \textit{Ашигласан материал, ном зүй} хэсэгт
bibitem командыг нэмнэ. \\


Жишээ нь: bibitem\{image1\} Гарчиг, Зохиогчдын нэр, хэвлэсэн он, хэвлэсэн газар

Дээрх жишээнд image1 гэдэг нь ишлэх нэр. Доод талын мөрөнд нь байгаа дарааллын дагуу
ашигласан материалыг бичнэ.

Ишлэхдээ cite командад ишлэх нэрийг дамжуулж өгнө. Жишээ нь cite\{image1\}.
\section{Зүүлт}
Зүүлтийг footnote командаар оруулна \footnote{Энэ холбоосоос зүүлтийн талаар дэлгэрэнгүй унш: \url{https://www.sharelatex.com/learn/Footnotes}}.

\chapter{Зураг}
Зураг оруулахдаа includegraphics командыг ашиглана. Доорх жишээнд figure01.png гэдэг нь зургийн файлын нэр бөгөөд өргөтгөлийг заавал бичих шаардлагагүй. Зургийн файл нь main.tex файлтай нэг фолдерт байх шаардлагатайг анхаарна уу! Дэлгэрэнгүйг \cite{image1}-с үз.

\includegraphics{figure01.png}


\section{Зургийн хэмжээ өөрчлөх}
Хэмжээг томруулахдаа 0-1 хооронд утга ашиглана. Хэрэв 2 гэвэл 2 дахин томроно.
\begin{center}
	includegraphics[scale=0.5]\{figure01\}
\end{center}

\includegraphics[scale=0.9]{figure01}

Өндөр өргөнийг шууд зааж өгч болох бөгөөд дөрвөлжин хаалтан дотор доорх байдлаар бичнэ.
\begin{center}
	includegraphics[width=3cm, height=4cm]\{figure01\}
\end{center}
\includegraphics[width=3cm, height=4cm]{figure01}

\section{Зураг эргүүлэх}
Зургийн эргүүлэхдээ angle параметрт эргүүлэх өнцгийн хэмжээг өгнө.
\begin{center}
	includegraphics[width=3cm, height=4cm, angle=45]\{figure01\}
\end{center}
\includegraphics[width=3cm, height=4cm, angle=45]{figure01}

\section{Зургийн нэр}
Зургын нэрийг begin\{figure\} хооронд includegraphics командтай хамт оруулна Зураг \ref{fig:lion1}-ыг хар.

Энд зургийн нэрээс гадна label-ийг давхар бичиж өгөх шаардлагатай ба энэ нь зургийн дугаараар заалт хийхэд ашиглана. Жишээ нь: Зураг \ref{fig:lion2}

\begin{figure}[h]
	\centering
	\includegraphics[scale=0.9]{figure01}
	\caption{Зураг голлуулах}
	\label{fig:lion1}
\end{figure}

\section{Зураг голлуулах}
Зургийг голлуулахдаа includegraphics командын өмнө centering
командыг бичээд reflectbox командыг includegraphics болон caption
командуудад үйлчлэхээр оруулна.

\begin{figure}[h]
	\includegraphics[scale=0.5]{figure01.png}
	\caption{Зургийн нэрийг энд бичнэ}

	\label{fig:lion2}
\end{figure}

\section{Зургийн чанар}
LaTex-т зургийг вектор форматаар (svg, eps) оруулбал хэвлэх болон томруулж харахад зургийн чанар
алдагдахгүй. Тиймээс аль болох вектор зураг оруулж өгвөл зүгээр.

\chapter{Хүснэгт оруулах}
Хүснэгт оруулахад tabular командыг ашигладаг \cite{table}.

\begin{table}[h]
	\centering
	\caption{Хүснэгтийн нэр. Хүснэгтийн нэр хүснэгтийн дээд талд байрлана. }
	\label{my-label}
	\begin{tabular}{|l|l|l|l|l|}
		\hline
		\textbf{Багана1} & \textbf{Багана2}  & \textbf{Багана3} & \textbf{Багана4} & \textbf{Багана5} \\ \hline
		өгөгдөл          & \textit{өгөгдөл1} &                  &                  &                  \\ \hline
		                 &                   &                  &                  &                  \\ \hline
		                 &                   &                  &                  &                  \\ \hline
	\end{tabular}
\end{table}

\section{Хүснэгт зурах хэрэгсэл}
Цэвэр LaTex кодоор Хүснэгт үүсгэхэд харьцангуй төвөгтэй байдаг учир
хялбар хэрэгслийг ашиглаж болно.

Тухайлбал \url{https://www.tablesgenerator.com/} холбоосруу орж хүснэгтийг визуал орчинд зураад үүсгэж өгсөн LaTex кодыг энд хуулж оруулна.

\chapter{Код ба алгоритм оруулах}
Код оруулахдаа begin\{lstlisting\}  ... end\{lstlisting\} командын хооронд бичнэ.

\begin{lstlisting}[language=C, caption=С хэлний кодын жишээ, frame=single]
#include <stdio.h>
#define N 10
/* Block
 * comment */
int main()
{
    int i;
    // Line comment.
    puts("Hello world!");
    for (i = 0; i < N; i++)
    {
        puts("LaTeX is also great for  programmers!");
    }
    return 0;
}
\end{lstlisting}

Мөн кодын эх файлыг шууд оруулж ирж болох бөгөөд доорх командыг бичнэ.

\lstinputlisting[language=C, firstline=11, lastline=16, caption=Кодын файлаас хэсэгчилж оруулах]{src/hello.c}

Мэдээллийн технологи, програм хангамжийн ажлын тайланд алгоримтыг хийсвэр кодын бичиглэлээр оруулах шаардлага гардаг. Дараах жишээгээр (Алгоритм \ref{alg:task_gen}) хийсвэр кодоор хэрхэн бичиж болохыг харуулав. Мөн бичвэр дотроо алгоритмд ашиглаж байгаа $parentId$ хувьсагчийг дурдаж бичиж болдог.

\makeatletter
\newenvironment{megaalgorithm}[1][htb]{%
	\renewcommand{\ALG@name}{Алгоритм}% Update algorithm name
	\begin{algorithm}[#1]%
		}{\end{algorithm}}
\makeatother

\begin{megaalgorithm}
	\caption{Даалгавар үүсгэх алгоритм}\label{alg:task_gen}
	\begin{algorithmic}[1]
		\Function{traverse}{$parentId$}\Comment{parentId--эцэг ойлголтын дугаар}
		\State $children \gets \Call{getChildConceptIds}{parentId$}
			\State $childCount \gets children.count$
		\If{$childCount == 0$}
		\State \textbf{return}
		\EndIf
		\For{$i = 0$ \textbf{to} $childCount$}
		\State \Call{generateTask}{$children_i$}\Comment{Орчуулгын даалгавар үүсгэх}
		\EndFor
		\For{$i = 0$ \textbf{to} $childCount$}
		\State \Call{traverse}{$children_i$}
		\EndFor
		\EndFunction
	\end{algorithmic}
\end{megaalgorithm}

%----------------------------------------------------------------------------------------
%   Дүгнэлт эндээс эхэлнэ
%----------------------------------------------------------------------------------------
\conclusion{Дүгнэлт}
сольсонДүгнэлтийг энд бич


%----------------------------------------------------------------------------------------
%   Дипломын номзүй, хавсралтын хэсэг эндээс эхэлнэ
%----------------------------------------------------------------------------------------

\singlespace
\addcontentsline{toc}{part}{НОМ ЗҮЙ}
\begin{thebibliography}{}
	% Ашигласан материалыг эндээс оруулна
	\bibitem{AES}
	Daemen, J., \& Rijmen, V. (2002). "The Design of Rijndael: AES - The Advanced Encryption Standard." Springer. p.1-2.
	\bibitem{pharagraph1}
	Paragraphs and new lines,  Share LaTex, \url{https://www.sharelatex.com/learn/Paragraphs_and_new_lines}
	\bibitem{intro_crypo}
	Д. Гармаа (2022). "Криптографын үндэс." Улаанбаатар хот.
	\bibitem{modern_crypto}
	Bellare, Mihir; Rogaway, Phillip (11 May 2005), Introduction to Modern Cryptography (Lecture notes), archived (PDF) from the original on 2023-10-30, chapter 3.

\end{thebibliography}


%----------------------------------------------------------------------------------------
%   Хавсралтууд эндээс эхэлнэ
%----------------------------------------------------------------------------------------
\appendix
\addcontentsline{toc}{part}{ХАВСРАЛТ}

% Хавсралтын нэр. Хавсралт гэдэг үг агуулахгүй
\chapter{Шинжилгээ зохиомж}
Хавсралтын агуулга

% Хавсралтын нэр. Хавсралт гэдэг үг агуулахгүй
\chapter{Кодын хэрэгжүүлэлт}

\begin{lstlisting}[language=Python]
	import numpy as np
	 
	def incmatrix(genl1,genl2):
			m = len(genl1)
			n = len(genl2)
			M = None #to become the incidence matrix
			VT = np.zeros((n*m,1), int)  #dummy variable
	 
			#compute the bitwise xor matrix
			M1 = bitxormatrix(genl1)
			M2 = np.triu(bitxormatrix(genl2),1) 
	 
			for i in range(m-1):
					for j in range(i+1, m):
							[r,c] = np.where(M2 == M1[i,j])
							for k in range(len(r)):
									VT[(i)*n + r[k]] = 1;
									VT[(i)*n + c[k]] = 1;
									VT[(j)*n + r[k]] = 1;
									VT[(j)*n + c[k]] = 1;
	 
									if M is None:
											M = np.copy(VT)
									else:
											M = np.concatenate((M, VT), 1)
	 
									VT = np.zeros((n*m,1), int)
	 
			return M
	\end{lstlisting}

\end{document}
