\begin{abstract}
	\setcounter{secnumdepth}{0}
	Энэхүү дипломын ажилд криптографын янз бүрийн алгоритм, программуудыг системтэйгээр судалсан бөгөөд үндсэн зорилго нь тэдгээрийн үндсэн бүтэц, үйл ажиллагааны механизм, практик хэрэглээг ойлгох явдал байв. Энэхүү судалгааны ажилд уламжлалт болон шинээр гарч ирж буй криптографын алгоритмуудыг судалж, гүйцэтгэл, аюулгүй байдал, үр ашигтай байдалд үндэслэн харьцуулсан судалгааг хийв.\\

	Энэхүү судалгаанд өгөгдлийн шифрлэлтийн стандарт (DES), дэвшилтэт шифрлэлтийн стандарт (AES), Ривест-Шамир-Адлеман (РСА (RSA)), эллиптик муруй криптографи (ECC) зэрэг тэгш хэмтэй болон тэгш хэмт бус криптографын алгоритмуудыг нарийвчлан судалсан.\\

	Төгсөлтийн ажлын практик хэсэгт хэд хэдэн криптографын программуудыг боловсруулж харьцуулсан ба орчин үеийн стандартыг хангасан тоон гарын үсгийн системийг үүлэн технологит суурилан бүтээсэн.
	\section{Зорилго}
	Энэхүү ажилд үүлэн технологит суурилсан тоон гарын үсгийн системийг бүтээж хэрэглэгчэд өөрсдийн цахим гарын үсгээр баталгаажсан файлуудыг интернэтэд хуваалцах боломжийг бүрдүүлэх гол зорилго зорилго тавьсан болно.
	\section{Зорилт}
	\begin{enumerate}
		\item Криптографын сонгодог алгоритмуудыг судлах, эзэмших
		\item Криптографын сонгодог алгоритмууд болон үүлэн технологид суурилсан тоон гарын үсгийн систем бүтээх
		\item Бүрэн бүтэн, хөндөгдөөгүй, эх сурвалжтай файлыг хуваалцах боломжийг бүрдүүлэх
	\end{enumerate}
	\section{Үндэслэл}
	Цахим харилцаа холбоо хурдацтай хөгжиж буй өнөөгийн нийгэмд, хуулийн дагуу хүчин төгөлдөр бичиг баримтыг интернэт сүлжээг ашиглан хуваалцах хэрэг байна. Гэсэн хэдий ч Монголд үүлэн дээр суурилсан тоон гарын үсгийн систем байхгүйгээс хэрэглэгчэд нийцгүй байгаа нь харагдаж байна.

	Дэлхийн банкны мэдээллээр Монгол Улсын иргэдийн дийлэнх хувь нь (2021 оны байдлаар 81.61\%) интернэт хэрэглэгч байгаа нь ийм системийн боломжит цар хүрээг харуулж байна. \footnote{Дэлхийн банкны судалгаа: \url{https://data.worldbank.org/indicator/IT.NET.USER.ZS?end=2021&locations=MN}}
	
	Түүнчлэн, одоо байгаа Клиент программууд нь Windows үйлдлийн системд зориулагдсан байдаг. Энэхүү Windows төвтэй арга нь нийцтэй байдлын асуудалд хүргэдэг. StatCounter Global Stats-аас гаргасан мэдээллээс харахад 2023 оны байдлаар дэлхий даяар үйлдлийн системийн зах зээлийн 30 орчим хувийг macOS болон Linux зэрэг Windows бус платформууд эзэлж байна.\footnote{Үйлдлийн системийн судалгаа: \url{https://gs.statcounter.com/os-market-share/desktop/worldwide}}
	
	Эдгээрийг авч үзвэл хэрэглэгчдийн олон талт хэрэгцээнд нийцсэн үүлэн технологит суурилсан тоон гарын үсгийн системийг хөгжүүлэх шаардлагатай байгаа нь харагдаж байна.

	%	\setcounter{secnumdepth}{0} reverse this command
	\setcounter{secnumdepth}{2}

\end{abstract}

