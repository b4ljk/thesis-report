\begin{abstract}
	\setcounter{secnumdepth}{0}
	Энэхүү дипломын ажилд криптографийн янз бүрийн алгоритм, программуудыг системтэйгээр судалсан бөгөөд үндсэн зорилго нь тэдгээрийн үндсэн бүтэц, үйл ажиллагааны механизм, практик хэрэглээг ойлгох явдал юм. Энэхүү судалгааны ажилд уламжлалт болон шинээр гарч ирж буй криптографийн алгоритмуудыг судалж, гүйцэтгэл, аюулгүй байдал, үр ашигтай байдалд үндэслэн харьцуулсан судалгааг хийв.\\

	Энэхүү судалгаанд өгөгдлийн шифрлэлтийн стандарт (DES), дэвшилтэт шифрлэлтийн стандарт (AES), Ривест-Шамир-Адлеман (RSA), эллиптик муруй криптографи (ECC) зэрэг тэгш хэмтэй болон тэгш бус криптограф алгоритмуудыг нарийвчлан судалсан.\\

	Төгсөлтийн ажлын практик хэсэгт хэд хэдэн криптографийн програмуудыг боловсруулж, харьцуулсан ба орчин үеийн стандартыг хангасан тоон гарын үсгийн системийг үүлэн технологид суурилан бүтээсэн.
	\section{Зорилго}
	Үүлэн технологид суурилсан тоон гарын үсгийн системийг бүтээснээр хэрэглэгчид өөрсдийн цахим гарын үсгээр баталгаажсан файлуудыг интернет хуваалцах боломжийг бүрдүүлэх гол зорилготой юм.
	\section{Зорилт}
	Бүрэн бүтэн байдал нь хөндөгдөөгүй, эх сурвалж нь тодорхой файлыг хуваалцах боломжийг бүрдүүлэх.
	\section{Үндэслэл}
	Монголд одоогийн байдлаар үүлэн технологид суурилсан тоон гарын үсгийн систем байхгүй байгаа нь хэрэглэгчид энэхүү технологийг ашиглахад төвөгтэй болгож байна.
	

\end{abstract}


