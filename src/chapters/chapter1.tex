\chapter{Онолын судалгаа}
\section{Тэгш хэмт крифтограф}
Тэгш хэмт крифтографт шифрлэлт болон шифр тайлах түлхүүрүүд адил байна. Тэгш хэмт алгоритм нь Тэгш бус хэмт шифрлэлтээс харьцангуй хурдан ажилдаг. Гэвч нууцалсан мэдээллийг тайлж унших түлхүүр болон нууцлах түлхүүр адилхан байх нь харилцагч талууд урьдчилан түлхүүрээ хоорондоо тохиролцох шаардлагыг гаргаж ирдэг. Энэ нь сул тал болох эрсдэлтэй. Хэрвээ гуравдагч этгээд түлхүүрийг олж авбал бүх нууцалсан мэдээллийг үзэх боломжтой болох юм.

Хамгийн түгээмэл хэрэглэгддэг тэгш хэмт шифрлэлтийн алгоритм бол Бельгийн криптографич Жоан Даемен, Винсент Рижмен нарын боловсруулсан Advanced Encryption Standard (AES) юм. AES нь хуучин Data Encryption Standard (DES)-ийг сольсон бөгөөд одоо дэлхий даяар ашиглагдаж байна.\cite{AES}
\subsection{Блокон шифрлэлт}

Хэрвээ эх ба шифрлэгдсэн тексүүдийн огторгуй нь ямар нэг $\sum_{}^{n}$ олонлог байвал тухайн криптографыг блокон шифрлэлт гэнэ. Блокон шифрлэлтэнд өгсөн мэдээг тэнцүү \textit{n} урттай хэсгүүдэд хуваан шифрлэдэг.\cite{intro_crypo}

Блок шифрт энгийн текстийн блокийг бүхэлд нь авч, шифрлэгдсэн текстийн блокыг үүсгэхэд ашигладаг. Блокийн хэмжээг ерөнхийдөө шифрийн алгоритмаар тодорхойлно. Ихэнх блок шифрүүдийн хувьд энэ нь ихэвчлэн 64 эсвэл 128 бит байдаг ба зарим тохиолдолд нууцлалыг нэмэх зорилгоор 256, 512 бит ч байж болдог.


Хоёр төрлийн алгоритм ашиглах ба нэг нь шифр хийхэд нөгөө нь тайлахад ашиглагддаг. Эдгээр нь \textit{n} урттай бит болон \textit{k} бит урттай түлхүүрийг авч \textit{n} бит урттай блок үүсгэнэ.\\$E: \{0,1\}^k \times \{0,1\}^n \rightarrow \{0,1\}^n$.
Тайлах алгоритм \textit{D}-г нууцлах функцын урвуу гэж тодорхойлж болно.\\ $D: \{0,1\}^k \times \{0,1\}^n \rightarrow \{0,1\}^n$\\
$\forall k \in \{0,1\}^k, \forall m \in \{0,1\}^n, D(k, E(k, m)) = m$\\
\cite{modern_crypto}

\subsection{Урсгалын шифрлэлт}
Урсгалын шифрлэлт гэдэг нь өгөгдлийг урсгал маягаар нэг дор нэг битийг Криптографын алгоритм болон түлхүүрээ ашиглан шифрлэх арга юм. Урсгалын шифрын давуу тал нь блок шифрлэлтээс харьцангуй хурдан ажиллахаас гадна, хэрэгжүүлэлтэнд бага код ордог билээ. Гэсэн хэдий ч орчин үед түгээмэл ашиглагдахаа больсон ба элдэв халдагад түгээмэл өртдөг нь үүнтэй холбоотой. Жишээ нь RC4 гэх Урсгалын шифрлэлтийн алгоритм нь WEB болон WPA хамгаалалтад ашиглагддаг байсан хэдий ч хангалттай сайн хамгаалалт болж чадахгүй байгаа тул, хэрэглээнээс халагдаж байна.

\section{Өгөгдөл шифрлэлтийн стандарт}
\subsection{DES алгоритм}
DES (Data Encryption Standard) нь 1970-аад онд хөгжүүлэгдсэн тэгш хэмт блок шифрлэлтийн алгоритм юм. DES нь 64 бит урттай блок дээр ажиллах ба үүнийг 32-бит урттай хоёр хэсэг $L_{0}, R_{0}$ болгон хувааж, баруун талын 32-бит урттай хэсгийг олон янзын аргаар хувиргаж эцэст нь $L_{0}$-тэй XOR үйлдэл хийнэ. Арван зургаан үе хувиргалтын дараагаар $L_{0}, R_{0}$ нийлүүлж 64 бит шифрлэгдсэн блокыг үүснэ.
\subsubsection{Чанарууд}

\subsection{AES-н үйлдлүүд}
\chapter{Системийн зохиомж}