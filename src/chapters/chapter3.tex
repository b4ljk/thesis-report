\chapter{Хэрэгжүүлэлт}
\section{Сонгосон технологи}
\subsection{Nextjs \& Reactjs}
\subsubsection{Declarative}
React нь хэрэглэгчийн интерактив интерфейс бүтээхийг хялбарчилдаг. Aппликейшны state бүрд зориулсан энгийн бүтэц зохион байгуулахаас гадна, React нь өгөгдөл өөрчлөгдөхөд яг зөв компонентоо өөрчлөн рендер хийдэг. Declarative бүтэц нь кодыг тань debug хийхэд хялбар болгохоос гадна, ажиллагаа нь илүү тодорхой болдог

\subsubsection{Компонент-д тулгуурласан}
Бие даан state-ээ удирддаг маш энгийн компонент бичиж, эдгээрийг хольж найруулан нарийн бүтэцтэй хэрэглэгчийн интерфейс бүтээ.

Компонентийн логик нь тэмплэйт-ээр бус JavaScript-ээр бичигддэг учраас өгөгдлийг апп хооронд хялбар дамжуулж, DOM-оос state-ээ тусд нь байлгаж чадна.

\subsubsection{Nextjs}
Netflix, TikTok, Hulu, Twitch, Nike гэсэн орчин үеийн аваргууд ашигладаг энэхүү орчин үеийн фрэймворк нь React технологи дээр үндэслэгдсэн бөгөөд Frontend Backend хоёр талд хоёуланд нь ажилладаг веб аппуудыг хийх чадвартайгаараа бусдаасаа давуу юм. Next.js -ийн үндсэн дизайн нь клиент болон сервер талын аль алиных давуу талыг ашиглаж чаддаг, ямар нэг дутагдалгүй веб сайтыг яаж хамгийн хурдан хялбар бүтээх вэ гэдгийг бодож тусгасан байдаг. Next.js нь сервер талд react компонентуудыг рендерлэн энгийн html, css, json файл болгон хувиргах замаар ажилладаг бөгөөд 2020 оноос олон нийтэд танигдсан JAMStack технологи болон статик сайт, автоматаар статик хуудас үүсгэх, CDN deployment, сервергүй функц, тэг тохиргоо, файлын системийн рүүтинг (PHP-ээс санаа авсан), SWR (stale while revalidate), сервер талд рендерлэх зэрэг асар олон орчин үеийн шинэхэн технологиудыг бүгдийг хийж чаддаг анхны бүрэн веб фрэймворк гэж хэлж болно.\cite{Reactjs}
\subsection{tRPC (Back End)}

Энгийнээр хэлбэл, tRPC нь клиент болон сервер хоорондоо сүлжээгээр харилцаж болох API (Application Programming Interfaces) бүтээх хэрэгсэл юм. Энэ нь хувьсагчийн төрлүүдийг нягт зааж өгч Front-End Back-End хоёрийг холбож ажилладаг. Жишээ нь хэрвээ сервер тал дээр ажиллаж байгаа хөгжүүлэгч, функцын параметр солиход энгийн REST api эсвэл Graphql түүнийг мэдэж чадахгүй юм. Харин tRPC нь шууд алдаа болж харагдах ба хөгжүүлэлтийн орчинд Back-end Front-end хоёр холбогдож ажилдаг гэдгээрээ давуу юм.

\begin{lstlisting}[language=Typescript, caption=tRPC тохиргоо, frame=single]
import { initTRPC, TRPCError } from "@trpc/server";
import { type CreateNextContextOptions } from "@trpc/server/adapters/next";
import { type Session } from "next-auth";
import superjson from "superjson";
import { ZodError } from "zod";

import { getServerAuthSession } from "~/server/auth";
import { db } from "~/server/db";

interface CreateContextOptions {
  session: Session | null;
}

const createInnerTRPCContext = (opts: CreateContextOptions) => {
  return {
    session: opts.session,
    db,
  };
};

export const createTRPCContext = async (opts: CreateNextContextOptions) => {
  const { req, res } = opts;

  // Get the session from the server using the getServerSession wrapper function
  const session = await getServerAuthSession({ req, res });

  return createInnerTRPCContext({
    session,
  });
};

const t = initTRPC.context<typeof createTRPCContext>().create({
  transformer: superjson,
  errorFormatter({ shape, error }) {
    return {
      ...shape,
      data: {
        ...shape.data,
        zodError:
          error.cause instanceof ZodError ? error.cause.flatten() : null,
      },
    };
  },
});

export const createTRPCRouter = t.router;
export const publicProcedure = t.procedure;

const enforceUserIsAuthed = t.middleware(({ ctx, next }) => {
  if (!ctx.session?.user) {
    throw new TRPCError({
      code: "UNAUTHORIZED",
      message: "Хэрэглэгч нэвтрээгүй байна".toUpperCase(),
    });
  }
  return next({
    ctx: {
      // infers the `session` as non-nullable
      session: { ...ctx.session, user: ctx.session.user },
    },
  });
});

export const protectedProcedure = t.procedure.use(enforceUserIsAuthed);

	\end{lstlisting}